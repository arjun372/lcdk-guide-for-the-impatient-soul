%% Abstract

% • Abstract: one sentence for each:
% – Introduction
% – Motivation
% – Approach
% – Results 

\begin{abstract}
%\boldmath
This paper discusses improvements made to the LCDK development workflow through a series of compute optimizations, implementation of I\textsuperscript{2}C, UART and GPIO communication protocols, and the introduction of Deep Neural Networks(DNN) exported from Tensorflow. We have been able to achieve between 5x-7.5x speedup in FFT execution times, 2x speedup in math ops, and 22x speed up in vectorized math ops. In addition, we have implemented support for 400kHz I\textsuperscript{2}C with bus-buffers, USB-UART, UART and GPIO. We have also implemented a 8-layer DNN that classifies human walking  "labels = {stationary, walking}" in real-time with F-measure 0.912 from 2000ms of wrist-worn accelerometer data sampled at 16 samples/sec and transferred over Bluetooth. We hope that these methodologies prove to be a valuable addition to the modern DSP engineer's toolkit.

\end{abstract}

%% Keywords

\begin{IEEEkeywords}
DSPLIB, MATHLIB, IMGLIB, I\textsuperscript{2}C, GPIO, UART, Tensorflow, Deep Learning, Deep Neural Network, Multilayer Perceptron, ReLU activation.
\end{IEEEkeywords}