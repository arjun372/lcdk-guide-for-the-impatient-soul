%% describe the shortcomings of existing systems (or your own knowledge of ot skill set in the subject) that motivated you to choose this project.

\section{Motivation}

The need for this project the arose due to a requirement for efficient and streamlined development tools which are otherwise inaccessible due to limited documentation or closed-off in a proprietary development environment setting. While other RISC based ARM platforms such Raspberry Pi or BeagleBone already provide much of the functionality developed here (including Tensorflow for ML and SciPy for DSP), they fail to provide low-level access and fine-grain control of hardware or even the sheer compute capability of the C6748 DSP. On the other hand, although some progress has been made in running Linux kernel on the LCDK, it is important to note that the Linux environment runs on the weaker ARM9 OMAPL138. Thus, any functionality gained by running Linux is immediately lost to sub-power performance due to lack of access to the DSP. It should be noted however, that the DSP can be enabled when running Linux, however this is not trivially achievable, and adds significant overhead to the system complexity. More information about enabling the DSP under Linux can be accessed here\cite{linux}.
